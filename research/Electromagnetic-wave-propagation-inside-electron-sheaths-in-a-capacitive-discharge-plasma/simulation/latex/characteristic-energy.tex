%This gives change in energy in terms of mass fraction

\frac{\Delta E}{E} = \frac{4 m M}{(m + M)^2}



%The average electron energy in a plasma, considering ballistic gain and collisional losses, is:

\epsilon \approx e E d \left(1 - N \cdot \frac{2 m_e}{M}\right)

%= e E \lambda \cdot \frac{1 - \left(1 - \frac{2 m_e}{M}\right)^{N+1}}{\frac{2 m_e}{M}}

\[
\epsilon_0 \approx \sqrt{\frac{3 m_e}{2 e} \cdot \frac{E/N}{\nu_m}}
\]

\[
\bar{\epsilon} = \frac{2}{3} \epsilon_0
\]


\text{Where:} \\
m_e \text{ = electron mass} \\
e \text{ = electron charge} \\
E/N \text{ = reduced electric field} \\
\nu_m \text{ = momentum transfer collision frequency}


%The characteristic energy \(\epsilon_0\) for a Druyvesteyn plasma, accounting for the mass fraction \(\frac{2 m_e}{M}\), is given by:

\[
\epsilon_0 = \frac{e^2 E^2}{m_e \nu_m^2} \cdot \frac{2 m_e}{M}
\]


%where \(e\) is the electron charge, \(E\) is the electric field, \(m_e\) is the electron mass, \(\nu_m\) is the collision frequency, and \(M\) is the neutral particle mass.

%The average electron energy is:

\[
\langle \epsilon \rangle = \frac{3}{2} \epsilon_0 = \frac{3 e^2 E^2 M}{4 m_e^2 \nu_m^2}.
\]


